\documentclass[11pt,reqno,letter]{amsart}

\usepackage{amsmath}
\usepackage{amsfonts}
\usepackage{graphicx}
%\usepackage{epstopdf}
\usepackage{hyperref}
\hypersetup{colorlinks=false}
\usepackage[left=1in,right=1in,top=0.9in,bottom=0.9in]{geometry}
\usepackage{multirow}
\usepackage{verbatim}
\usepackage{fancyhdr}
\usepackage{mdframed}
%\usepackage[small,compact]{titlesec} 
\usepackage{listings}
\usepackage{ccicons}

\usepackage{natbib}
\renewcommand{\cite}{\citet}

%\usepackage{pxfonts}
%\usepackage{isomath}
%\usepackage{mathpazo}
\usepackage{newpxmath}
\usepackage{newpxtext}
%\usepackage{arev} %     (Arev/Vera Sans)
%\usepackage{eulervm} %_   (Euler Math)
%\usepackage{fixmath} %  (Computer Modern)
%\usepackage{hvmath} %_   (HV-Math/Helvetica)
%\usepackage{tmmath} %_   (TM-Math/Times)
%\usepackage{cmbright}
%\usepackage{ccfonts} \usepackage[T1]{fontenc}
%\usepackage[garamond]{mathdesign}
\usepackage{color}
\usepackage[normalem]{ulem}

\newtheorem{theorem}{Theorem}[section]
\newtheorem{conjecture}{Conjecture}[section]
\newtheorem{corollary}{Corollary}[section]
\newtheorem{lemma}{Lemma}[section]
\newtheorem{proposition}{Proposition}[section]
\theoremstyle{definition}
\newtheorem{assumption}{}[section]
%\renewcommand{\theassumption}{C\arabic{assumption}}
\newtheorem{definition}{Definition}[section]
\newtheorem{step}{Step}[section]
\newtheorem{remark}{Comment}[section]
\newtheorem{example}{Example}[section]
\newtheorem*{example*}{Example}

\linespread{1.0}

\pagestyle{fancy}
%\renewcommand{\sectionmark}[1]{\markright{#1}{}}
\fancyhead{}
\fancyfoot{} 
%\fancyhead[LE,LO]{\tiny{\thepage}}
\fancyhead[CE,CO]{\tiny{\rightmark}}
\fancyfoot[C]{\small{\thepage}}
\renewcommand{\headrulewidth}{0pt}
\renewcommand{\footrulewidth}{0pt}

\fancypagestyle{plain}{%
\fancyhf{} % clear all header and footer fields
\fancyfoot[C]{\small{\thepage}} % except the center
\renewcommand{\headrulewidth}{0pt}
\renewcommand{\footrulewidth}{0pt}}

\makeatletter
\renewcommand{\@maketitle}{
  \null
  \begin{center}%
    \rule{\linewidth}{1pt} 
    {\Large \textbf{\textsc{\@title}}} \par
    {\normalsize \textsc{Paul Schrimpf}} \par
    {\normalsize \textsc{\@date}} \par
    {\small \textsc{University of British Columbia}} \par
    {\small \textsc{Economics 326}} \par
    \ccbysa\footnote{This work is licensed under a 
      \href{http://creativecommons.org/licenses/by-sa/4.0/}{Creative
        Commons Attribution-ShareAlike 4.0 International License}}
    \par

    \rule{\linewidth}{1pt} 
  \end{center}%
  \par \vskip 0.9em
}
\makeatother

\newcommand{\argmax}{\operatornamewithlimits{arg\,max}}
\newcommand{\argmin}{\operatornamewithlimits{arg\,min}}
\def\inprobLOW{\rightarrow_p}
\def\inprobHIGH{\,{\buildrel p \over \rightarrow}\,} 
\def\inprob{\,{\inprobHIGH}\,} 
\def\indist{\,{\buildrel d \over \rightarrow}\,} 
\def\F{\mathbb{F}}
\def\R{\mathbb{R}}
\newcommand{\gmatrix}[1]{\begin{pmatrix} {#1}_{11} & \cdots &
    {#1}_{1n} \\ \vdots & \ddots & \vdots \\ {#1}_{m1} & \cdots &
    {#1}_{mn} \end{pmatrix}}
\newcommand{\iprod}[2]{\left\langle {#1} , {#2} \right\rangle}
\newcommand{\norm}[1]{\left\Vert {#1} \right\Vert}
\newcommand{\abs}[1]{\left\vert {#1} \right\vert}
\renewcommand{\det}{\mathrm{det}}
\newcommand{\rank}{\mathrm{rank}}
\newcommand{\spn}{\mathrm{span}}
\newcommand{\row}{\mathrm{Row}}
\newcommand{\col}{\mathrm{Col}}
\renewcommand{\dim}{\mathrm{dim}}
\newcommand{\prefeq}{\succeq}
\newcommand{\pref}{\succ}
\newcommand{\seq}[1]{\{{#1}_n \}_{n=1}^\infty }
\renewcommand{\to}{{\rightarrow}}
\providecommand{\En}{\mathbb{E}_n}
\providecommand{\Er}{{\mathrm{E}}}
\providecommand{\var}{{\mathrm{Var}}}
\providecommand{\cov}{{\mathrm{Cov}}}
\providecommand{\corr}{{\mathrm{Corr}}}
\providecommand{\svar}{{\widehat{\var}}}
\providecommand{\scov}{{\widehat{\cov}}}
\renewcommand{\Pr}{{\mathrm{P}}}
\providecommand{\set}[1]{\left\{#1\right\}}
\providecommand{\plim}{\operatornamewithlimits{plim}}
\newcommand\indep{\protect\mathpalette{\protect\independenT}{\perp}}
\def\independenT#1#2{\mathrel{\setbox0\hbox{$#1#2$}%
    \copy0\kern-\wd0\mkern4mu\box0}} 
\providecommand{\avg}{{\frac{1}{n} \sum_{i=1}^n}}
\providecommand{\sumin}{{\sum_{i=1}^n}}

\providecommand{\bols}{{\hat{\beta}^{\mathrm{OLS}}}}
\providecommand{\biv}{{\hat{\beta}^{\mathrm{IV}}}}
\providecommand{\btsls}{{\hat{\beta}^{\mathrm{2SLS}}}}

\newtheoremstyle{problem}% name
{12pt}% Space above
{5pt}% Space below
{}% Body font
{}% Indent amount
{\bfseries}% Theorem head font
{:}% Punctuation after theorem head
{.5em}% Space after theorem head
{}% Theorem head spec (can be left empty, meaning `normal')

\theoremstyle{problem}
\newtheorem{problem}{Problem}

\newtheoremstyle{solution}% name
{2pt}% Space above
{12pt}% Space below
{}% Body font
{}% Indent amount
{\bfseries}% Theorem head font
{:}% Punctuation after theorem head
{.5em}% Space after theorem head
{}% Theorem head spec (can be left empty, meaning `normal')
\newtheorem{soln}{Solution}

\newenvironment{solution}
  {\begin{mdframed}\begin{soln}$\,$}
  {\end{soln}\end{mdframed}}



\lstset{language=R}
\lstset{keywordstyle=\color[rgb]{0,0,1},                                        % keywords
        commentstyle=\color[rgb]{0.133,0.545,0.133},    % comments
        stringstyle=\color[rgb]{0.627,0.126,0.941}      % strings
}       
%\lstset{
%  showstringspaces=false,       % not emphasize spaces in strings 
%  columns=fixed,
%  numbersep=3mm, numbers=left, numberstyle=\tiny,       % number style
%  frame=none,
%  framexleftmargin=6mm, xleftmargin=6mm         % tweak margins
%}

\definecolor{lightgray}{gray}{0.95}
\lstset{
  backgroundcolor=\color{lightgray},  
  showstringspaces=false,       % not emphasize spaces in strings 
  columns=fixed,
  frame = single,
  %numbersep=3mm, numbers=left, numberstyle=\tiny,       % numberstyle
  basicstyle=\footnotesize\ttfamily
  %frame=1mm,
  %framexleftmargin=4mm, xleftmargin=4mm         % tweak margins
}


\title{Economics 628 : Assignment 1}
\author{Paul Schrimpf}
\date{Due: September 12th, 2018}

\begin{document}
\maketitle

\section{Theory questions}

\begin{problem}
  In the matching model from section 1.2 of the notes, let $p_0(x) =
  \Pr(d=1|x)$ and $\mu(a,x) = \Er[y|d=a,x]$. 
  \begin{enumerate}
  \item Let
    \begin{align*}
      \theta^a = & \Er\left[\frac{y_i d_i}{p(x_i)} + \frac{y_i
               (1-d_i)}{1-p(x_i)} \right] \\
      \theta^b = & \Er\left[\mu(1,x_i) - \mu(0,x_i)\right] \\
      \theta^c = & \Er\left[d_i \frac{y_i -\mu(1,x_i)}{p(x_i)} +
                   (1-d_i)\frac{y_i - \mu(0,x_i)}{1-p(x_i)} +
                   \mu(1,x_i) - \mu(0,x_i)\right]
    \end{align*}
    Show that $\theta^a = \theta^b = \theta^c = \Er[y(1) - y(0)]$
  \item Following the notation from section 1.6, let
    \begin{align*}
      \psi^a(w_i;\theta,p,\mu) = & \theta - \left[\frac{y_i d_i}{p(x_i)} + \frac{y_i
                                   (1-d_i)}{1-p(x_i)} \right] \\
      \psi^b(w_i;\theta,p,\mu) = & \theta - \left[\mu(1,x_i) - \mu(0,x_i)\right] \\
      \psi^c(w_i;\theta,p,\mu) = & \theta - \left[d_i \frac{y_i -\mu(1,x_i)}{p(x_i)} +
                                   (1-d_i)\frac{y_i - \mu(0,x_i)}{1-p(x_i)} +
                                   \mu(1,x_i) - \mu(0,x_i)\right]
    \end{align*}                                   
    Calculate the Fr\'{e}chet (directional) derivatives of
    $\Er[\psi^a(w_i;\theta,p,\mu)]$ with respect to $p$ and $\mu$ at
    $p_0, \mu_0$. Do the same for $\psi^b$ and $\psi^c$.    
  \end{enumerate}
\end{problem}


\begin{problem}
  Consider the following IV model:
  \begin{align*}
    y_i = \theta d_i + \epsilon_i \\
    d_i = f(z_i) + u_i
  \end{align*}
  with $\Er[\epsilon|z] = 0$ and $\Er[u|z] = 0$. Let $\hat{f}()$ be an
  estimate of $f$ that satisfies the following high level assumptions:
  \begin{align}
    \En[d_i \hat{f}(x_i)] - \En[d_i \hat{f}(x_i)] = & O_p(n^{-1/2} +
                                                      r_n) \label{r1} \\
    \En[(\hat{f}(x_i) - f(x_i)) \epsilon_i] = & O_p(n^{-1/2} r_n) \label{r1}
  \end{align}
  for some $r_n \to 0$. Show that the estimate
  \begin{align*}
    \hat{\theta} = \En[\hat{f}(x_i) d_i]^{-1} \En[\hat{f}(x_i) y_i ]
  \end{align*}
  is $\sqrt{n}$ asymptotically normal. State any additional
  assumptions needed. {\slshape{Hint: follow the steps of section
      1.6.1 in the notes}.}
\end{problem}

\section{Empirical questions}

\begin{problem} $\;$
  \url{http://tryr.codeschool.com/} is an interactive introduction to
  R. Please work through it if you have not used R before. If you're
  already familiar with R, then you can skip this.

  If you dislike tryR, alternative introductory resources include
  chapter 2.3 of \cite{james2013} and \href{Swirl}
  {https://swirlstats.com/students.html}.   
\end{problem}

If you're new to R, here is some advice about additional tools to
use. You can download R from
\href{https://cran.r-project.org/}{CRAN}.\footnote{You might also consider the
version of R distribution by Microsoft,
\href{https://mran.microsoft.com/rro}{Microsoft R Open}. The main
benefit is that Microsoft R Open uses Intel's Math Kernel Library for
linear algebra, which is quite fast. It's also possible to use Intel's
MKL or another high performance BLAS \& Lapack implementation (like
openBLAS) witah the plain version of R and get similar
performance.}
R itself comes with either no GUI nor text-editor (on Linux) or a
basic GUI with a limited text editor, (on Windows and I'd guess on
Mac, but I don't know). There are numerous programs that provide a
nicer way of working with R. The most popular is
\href{https://www.rstudio.com/} {RStudio}. It gives nice syntax
highlighting, easier debugging, etc; it is somewhat similar to
Matlab's GUI.  

A potentially useful, but not essential, tool for writing assignments
is the \href{http://rmarkdown.rstudio.com/}{Rmarkdown package}. It
lets you combine R code and text into a single document and produces
nice looking output in multiple formats. I'm using Rmarkdown to create
the slides and notes for this portion of the course. For research, I
often use Rmarkdown for preliminary data work that I'm just looking at
or sharing with coauthors and still making changes frequently.

\begin{problem}
  Explore the predictive performance of machine learning in your
  favorite dataset. Follow steps similar to the examples in section
  2.1 of the notes. Fit OLS, Lasso, random forest, and optionally
  additional machine learning estimators using a randomly chosen
  training subset of your data. Then create table(s) and/or figure(s)
  comparing the performance of the estimators.

  Suggested steps:
  \begin{enumerate}
  \item \textbf{Load the data and needed packages:}
    
    If your favorite dataset happens to already be in R format you can
    load it into R with simply,
\begin{lstlisting}
load("favoriteData.Rdata")
ls() # print all objects in current environment, to find out the 
     # name of the dataframe and other object in "favoriteData.Rdata"
\end{lstlisting}
    More likely, your dataset will be in another format, perhaps csv
\begin{lstlisting}
data <- read.csv("favoriteData.csv") # you may need to add some options
                                     # enter "?read.csv" for details
\end{lstlisting}
    or Stata,
\begin{lstlisting}
library(foreign)  # you will have to install this package with
                  # 'install.packages("foreign")' first
data <- read.dta("favoriteData.dta")
\end{lstlisting}
    You should check that your data has been read correctly and contains
    what you expect by looking at some summary statistics
\begin{lstlisting}
summary(data)
\end{lstlisting}
    We'll use the ``glmnet'' package for Lasso and ``grf'' for random
    forests. We might as well as also install ``ggplot2'' for plotting
    later. Install them with
\begin{lstlisting}
install.packages(c("glmnet","grf","ggplot2")) 

# load them
library(glmnet)
library(grf)
library(ggplot2)
\end{lstlisting}

  \item \textbf{OLS:} Suppose you have a variable named ``outcome''
    that you want to predict with variables names ``x1'', ``x2'', and
    ``x3''. You can do the following   
\begin{lstlisting}
# missing values will create size mismatches later. there are other
# workarounds (see na.action), but let's just drop them 
data <- subset(data, rowSums(is.na(data[,c("outcome","x1","x2","x3")]))==0)

train <- runif(nrow(data))<0.5 # use half data for training

# estimate ols on training sample
ols <- lm(outcome ~ x1 + x2 + x3, data=subset(data, train),
          x=TRUE, y=TRUE) # we want the output to include the x and y
                          # matrices
data$y.hat.ols <- predict(ols, newdata=data) # make predictions for whole
                                             # sample
# compute mse and mae for training and holdout samples
by(data, train, FUN=function(df) {
  err <- with(df, y.hat.ols-outcome)
  out <- c(mean(err), mean(err^2), mean(abs(err)))
  names(out) <- c("mean error","MSE","MAE")
  return(out)
})
\end{lstlisting}
  \item \textbf{Random forest:} to estimate a random forest and get predictions:
\begin{lstlisting}
Xt <- ols$x[,2:ncol(ols$x)] # we don't want the column of 1's 
yt <- ols$y
rf <- regression_forest(Xt,yt,tune.parameters=TRUE)
# for prediction, we need X for the whole sample, so
olsall <- lm(outcome ~ x1 + x2 + x3, data=data, x=TRUE) 
X <- olsall$x[,2:ncol(olsall$x)]
y.hat.rf <- predict(rf, X)$predictions
\end{lstlisting}
    You can then calculate MSE and other statistics the same as in the
    OLS example.
    
  \item \textbf{Lasso:} to estimate a Lasso model and get predictions,
    you should first expand your X matrix to have more variables. If
    you data includes many more potential regressors, you can just
    include them. If not, you can add interactions, powers, and other
    transformations of the original regressors. The following includes
    a 2nd degree polynomial in the 3 regressors and a linear spline in
    each with knots at -1, 0, and 1.\footnote{This choice is completely
    arbitrary and is not meant as a good suggestion. Instead, it is
    meant to illustrute some of the features of R's formulas. See the
    code for the pipeline example in the notes for a more practical
    specification.} 
\begin{lstlisting}
bigreg <- lm(outcome ~ polym(x1, x2, x3,degree=2) +
               (I(x1>-1) + I(x2>-1) + I(x3>-1) +
                I(x1>0) + I(x2>0) + I(x3>0) +
                I(x1>1) + I(x2>1) + I(x3>1))*(x1+x2+x3),
             data=data)
Xlasso <- Xlasso <- bigreg$x[,2:ncol(bigreg$x)]
lasso <- cv.glmnet(Xlasso[train, ], yt, alpha=1,
                   standardize=TRUE, intercept=TRUE)
data$y.hat.lasso <- predict(lasso, Xlasso, s=lasso$lambda.min,
                            type="response")
\end{lstlisting}

  \item If you want, you could create plots of the densities of errors
    or the prediction vs actual outcome, as in the notes. 
  \end{enumerate}
\end{problem}

\bibliographystyle{jpe}
\bibliography{/home/paul/ECON628_2018/paul/01-machineLearningAndCausalInference/ml.bib}

\end{document}
