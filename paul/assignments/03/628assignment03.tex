\documentclass[11pt,reqno,letter]{amsart}

\usepackage{amsmath}
\usepackage{amsfonts}
\usepackage{graphicx}
%\usepackage{epstopdf}
\usepackage{hyperref}
\hypersetup{colorlinks=false}
\usepackage[left=1in,right=1in,top=0.9in,bottom=0.9in]{geometry}
\usepackage{multirow}
\usepackage{verbatim}
\usepackage{fancyhdr}
\usepackage{mdframed}
%\usepackage[small,compact]{titlesec} 
\usepackage{listings}
\usepackage{ccicons}

\usepackage{natbib}
\renewcommand{\cite}{\citet}

%\usepackage{pxfonts}
%\usepackage{isomath}
%\usepackage{mathpazo}
\usepackage{newpxmath}
\usepackage{newpxtext}
%\usepackage{arev} %     (Arev/Vera Sans)
%\usepackage{eulervm} %_   (Euler Math)
%\usepackage{fixmath} %  (Computer Modern)
%\usepackage{hvmath} %_   (HV-Math/Helvetica)
%\usepackage{tmmath} %_   (TM-Math/Times)
%\usepackage{cmbright}
%\usepackage{ccfonts} \usepackage[T1]{fontenc}
%\usepackage[garamond]{mathdesign}
\usepackage{color}
\usepackage[normalem]{ulem}

\newtheorem{theorem}{Theorem}[section]
\newtheorem{conjecture}{Conjecture}[section]
\newtheorem{corollary}{Corollary}[section]
\newtheorem{lemma}{Lemma}[section]
\newtheorem{proposition}{Proposition}[section]
\theoremstyle{definition}
\newtheorem{assumption}{}[section]
%\renewcommand{\theassumption}{C\arabic{assumption}}
\newtheorem{definition}{Definition}[section]
\newtheorem{step}{Step}[section]
\newtheorem{remark}{Comment}[section]
\newtheorem{example}{Example}[section]
\newtheorem*{example*}{Example}

\linespread{1.0}

\pagestyle{fancy}
%\renewcommand{\sectionmark}[1]{\markright{#1}{}}
\fancyhead{}
\fancyfoot{} 
%\fancyhead[LE,LO]{\tiny{\thepage}}
\fancyhead[CE,CO]{\tiny{\rightmark}}
\fancyfoot[C]{\small{\thepage}}
\renewcommand{\headrulewidth}{0pt}
\renewcommand{\footrulewidth}{0pt}

\fancypagestyle{plain}{%
\fancyhf{} % clear all header and footer fields
\fancyfoot[C]{\small{\thepage}} % except the center
\renewcommand{\headrulewidth}{0pt}
\renewcommand{\footrulewidth}{0pt}}

\makeatletter
\renewcommand{\@maketitle}{
  \null
  \begin{center}%
    \rule{\linewidth}{1pt} 
    {\Large \textbf{\textsc{\@title}}} \par
    {\normalsize \textsc{Paul Schrimpf}} \par
    {\normalsize \textsc{\@date}} \par
    {\small \textsc{University of British Columbia}} \par
    {\small \textsc{Economics 326}} \par
    \ccbysa\footnote{This work is licensed under a 
      \href{http://creativecommons.org/licenses/by-sa/4.0/}{Creative
        Commons Attribution-ShareAlike 4.0 International License}}
    \par

    \rule{\linewidth}{1pt} 
  \end{center}%
  \par \vskip 0.9em
}
\makeatother

\newcommand{\argmax}{\operatornamewithlimits{arg\,max}}
\newcommand{\argmin}{\operatornamewithlimits{arg\,min}}
\def\inprobLOW{\rightarrow_p}
\def\inprobHIGH{\,{\buildrel p \over \rightarrow}\,} 
\def\inprob{\,{\inprobHIGH}\,} 
\def\indist{\,{\buildrel d \over \rightarrow}\,} 
\def\F{\mathbb{F}}
\def\R{\mathbb{R}}
\newcommand{\gmatrix}[1]{\begin{pmatrix} {#1}_{11} & \cdots &
    {#1}_{1n} \\ \vdots & \ddots & \vdots \\ {#1}_{m1} & \cdots &
    {#1}_{mn} \end{pmatrix}}
\newcommand{\iprod}[2]{\left\langle {#1} , {#2} \right\rangle}
\newcommand{\norm}[1]{\left\Vert {#1} \right\Vert}
\newcommand{\abs}[1]{\left\vert {#1} \right\vert}
\renewcommand{\det}{\mathrm{det}}
\newcommand{\rank}{\mathrm{rank}}
\newcommand{\spn}{\mathrm{span}}
\newcommand{\row}{\mathrm{Row}}
\newcommand{\col}{\mathrm{Col}}
\renewcommand{\dim}{\mathrm{dim}}
\newcommand{\prefeq}{\succeq}
\newcommand{\pref}{\succ}
\newcommand{\seq}[1]{\{{#1}_n \}_{n=1}^\infty }
\renewcommand{\to}{{\rightarrow}}
\providecommand{\En}{\mathbb{E}_n}
\providecommand{\Er}{{\mathrm{E}}}
\providecommand{\var}{{\mathrm{Var}}}
\providecommand{\cov}{{\mathrm{Cov}}}
\providecommand{\corr}{{\mathrm{Corr}}}
\providecommand{\svar}{{\widehat{\var}}}
\providecommand{\scov}{{\widehat{\cov}}}
\renewcommand{\Pr}{{\mathrm{P}}}
\providecommand{\set}[1]{\left\{#1\right\}}
\providecommand{\plim}{\operatornamewithlimits{plim}}
\newcommand\indep{\protect\mathpalette{\protect\independenT}{\perp}}
\def\independenT#1#2{\mathrel{\setbox0\hbox{$#1#2$}%
    \copy0\kern-\wd0\mkern4mu\box0}} 
\providecommand{\avg}{{\frac{1}{n} \sum_{i=1}^n}}
\providecommand{\sumin}{{\sum_{i=1}^n}}

\providecommand{\bols}{{\hat{\beta}^{\mathrm{OLS}}}}
\providecommand{\biv}{{\hat{\beta}^{\mathrm{IV}}}}
\providecommand{\btsls}{{\hat{\beta}^{\mathrm{2SLS}}}}

\newtheoremstyle{problem}% name
{12pt}% Space above
{5pt}% Space below
{}% Body font
{}% Indent amount
{\bfseries}% Theorem head font
{:}% Punctuation after theorem head
{.5em}% Space after theorem head
{}% Theorem head spec (can be left empty, meaning `normal')

\theoremstyle{problem}
\newtheorem{problem}{Problem}

\newtheoremstyle{solution}% name
{2pt}% Space above
{12pt}% Space below
{}% Body font
{}% Indent amount
{\bfseries}% Theorem head font
{:}% Punctuation after theorem head
{.5em}% Space after theorem head
{}% Theorem head spec (can be left empty, meaning `normal')
\newtheorem{soln}{Solution}

\newenvironment{solution}
  {\begin{mdframed}\begin{soln}$\,$}
  {\end{soln}\end{mdframed}}



\lstset{language=R}
\lstset{keywordstyle=\color[rgb]{0,0,1},                                        % keywords
        commentstyle=\color[rgb]{0.133,0.545,0.133},    % comments
        stringstyle=\color[rgb]{0.627,0.126,0.941}      % strings
}       
%\lstset{
%  showstringspaces=false,       % not emphasize spaces in strings 
%  columns=fixed,
%  numbersep=3mm, numbers=left, numberstyle=\tiny,       % number style
%  frame=none,
%  framexleftmargin=6mm, xleftmargin=6mm         % tweak margins
%}

\definecolor{lightgray}{gray}{0.95}
\lstset{
  backgroundcolor=\color{lightgray},  
  showstringspaces=false,       % not emphasize spaces in strings 
  columns=fixed,
  frame = single,
  %numbersep=3mm, numbers=left, numberstyle=\tiny,       % numberstyle
  basicstyle=\footnotesize\ttfamily
  %frame=1mm,
  %framexleftmargin=4mm, xleftmargin=4mm         % tweak margins
}


\title{Economics 628 : Assignment 3}
\author{Paul Schrimpf}
\date{Due: October 3rd, 2018}

\begin{document}
\maketitle

Complete one of the following problems. These questions are
intentionally somewhat open-ended, and one could spend a very long
time working on them. If after 5 hours or so, you are not nearing a
complete solution, it is okay to simply describe how far you have
gotten and what difficulties you encountered. 

\begin{problem}$\;$
  Using data from a randomized experiment, conduct a similar exercise
  as in the slides on the PKH experiment.
  \begin{itemize}
  \item Reproduce the main results.
  \item Estimate group average treatment effects and best linear
    predictions of the conditional average treatment effect using
    machine learning proxies.
  \item If notable treatment heterogeneity is found, investigate the
    means of covariates conditional on groups.
  \item Discuss your findings.
  \end{itemize}
  Data from many randomized experiments can be
  found on journal websites. If you do not want to search for one
  (which admittedly is time-consuming), there are a few experimental
  datasets at \url{https://github.com/gsbDBI/ExperimentData}.
  Also, the classic Lalonde data on the NSW  training program is
  available at  
  \url{http://users.nber.org/~rdehejia/data/nswdata2.html}. 
\end{problem}

\begin{problem}
  For a semiparametric model (that we have not discussed in class
  and is not used as an example in \cite{chernozhukov2018}) derive a double
  debiased machine learning estimator. If you can't come up with an
  appropriate model, \cite{chen2007} begins with a few examples, or
  feel free to ask for suggestions.
  \begin{itemize}
  \item Describe the model and its original moment conditions.
  \item Derive an orthogonal moment condition.
  \item Either:
    \begin{itemize}
    \item State primitive conditions on your model, data, and estimator
      for nuisance parameters such that the conditions of theorem 3.1 of
      \cite{chernozhukov2018}. See sections 4 and 5 of
      \cite{chernozhukov2018} for examples of what I mean. As remarks
      4.1 and 5.1 say, the rate conditions stated in the paper are
      ``non-primative.'' Replace them with primitive conditions. For
      example, a more precise version of ``The true function being
      estimated is approximately sparse with sparsity index satisfying
      some rate condition. We estimate with Lasso with penalty chosen in
      some specific way.'' OR
    \item Simulate the model and estimation method. Compare either the
      performance of different machine learning methods for estimating
      the nuisance parameters (like I did in the notes for the
      partially linear model) and/or estimation based on the
      orthogonal or non-orthogonal moment condition.
    \end{itemize}
  \end{itemize}
\end{problem}



%%%%%%%%%%%%%%%%%%%%%%%%%%%%%%%%%%%%%%%%%%%%%%%%%%


\bibliographystyle{jpe}
\bibliography{/home/paul/ECON628_2018/paul/01-machineLearningAndCausalInference/ml.bib}

\end{document}
